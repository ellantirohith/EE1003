\let\negmedspace\undefined
\let\negthickspace\undefined
\documentclass[journal,12pt,onecolumn]{IEEEtran}
\usepackage{cite}
\usepackage{amsmath,amssymb,amsfonts,amsthm}
\usepackage{amsmath}
\usepackage{algorithmic}
\usepackage{graphicx}
\usepackage{textcomp}
\usepackage{xcolor}
\usepackage{txfonts}
\usepackage{listings}
\usepackage{multicol}
\usepackage{enumitem}
\usepackage{mathtools}
\usepackage{gensymb}
\usepackage{comment}
\usepackage[breaklinks=true]{hyperref}
\usepackage{listings}
\usepackage{gvv}                                        
\usepackage[latin1]{inputenc}                                
\usepackage{color}                                            
\usepackage{array}                                            
\usepackage{longtable}                                       
\usepackage{calc}                                             
\usepackage{multirow}                                         
\usepackage{hhline}                                           
\usepackage{ifthen}                                           
\usepackage{lscape}
\usepackage{tabularx}
\usepackage{array}
\usepackage{float}
\usepackage{tikz}
\usepackage{multicol}
\usepackage{circuitikz}
\usetikzlibrary{patterns}
\newtheorem{theorem}{Theorem}[section]
\newtheorem{problem}{Problem}
\newtheorem{proposition}{Proposition}[section]
\newtheorem{lemma}{Lemma}[section]
\newtheorem{corollary}[theorem]{Corollary}
\newtheorem{example}{Example}[section]
\newtheorem{definition}[problem]{Definition}
\newcommand{\BEQA}{\begin{eqnarray}}
\newcommand{\EEQA}{\end{eqnarray}}
\theoremstyle{remark}

\begin{document}

\bibliographystyle{IEEEtran}
\vspace{3cm}

\title{Scientific Calculator}
\author{EE24BTECH11020    Ellanti Rohith}
\maketitle

\renewcommand{\thefigure}{\theenumi}
\renewcommand{\thetable}{\theenumi}



\section{Introduction}
This report details the implementation of a scientific calculator using an Arduino Uno, a 16x2 LCD display, and a 5x5 matrix keypad. The calculator supports:
\begin{enumerate}
    \item Standard arithmetic operations (+,-, *, /)
    \item Trigonometric functions (sin, cos, tan)
    \item Logarithmic calculations (log, ln)
    \item Exponentiation and square root
    \item Factorial calculations
    \item Modular arithmetic
    \item Shift functionality for inverse trigonometric functions, and mathematical constants.
\end{enumerate}

\section{Hardware Components}
\begin{enumerate}
    \item \textbf{Arduino Uno} - Microcontroller for processing inputs and outputs
    \item \textbf{16x2 LCD Display} - Displays user input and results
    \item \textbf{5x5 Button Matrix Keypad} - Used for user input (numbers, functions, operations)
    \item \textbf{10k$\Omega$ Potentiometer} - Adjusts LCD contrast
    \item \textbf{Breadboard and Jumper Wires} - For making circuit connections
    \item \textbf{Power Supply} - Arduino powered via USB or external 5V source
\end{enumerate}

\section{Circuit Connections}

\subsection{LCD to Arduino Connections}
The \textbf{16x2 LCD Display} is connected to the \textbf{Arduino Uno} in \textbf{4-bit mode}, using six digital/analog pins for communication.

\begin{center}
\begin{tabular}{|c|c|c|}
\hline
\textbf{LCD Pin} & \textbf{Arduino Pin} & \textbf{Function} \\
\hline
VSS & GND & Ground \\
VDD & +5V & Power Supply \\
V0 & Potentiometer (10k$\Omega$) & Contrast Adjustment \\
RS & D12 & Register Select \\
RW & GND & Always set to Ground (Write mode) \\
E & D13 & Enable Signal \\
D4 & A0 & Data Line 4 \\
D5 & A1 & Data Line 5 \\
D6 & A2 & Data Line 6 \\
D7 & A3 & Data Line 7 \\
A (LED +) & +5V & LCD Backlight Positive \\
K (LED -) & GND & LCD Backlight Negative \\
\hline
\end{tabular}
\end{center}

The \textbf{10k$\Omega$ potentiometer} is used to adjust the contrast of the LCD. One terminal of the potentiometer is connected to +5V, the other to GND, and the middle pin is connected to V0 on the LCD.

\subsection{Keypad to Arduino Connections}
The \textbf{5x5 keypad matrix} consists of \textbf{5 rows (outputs)} and \textbf{5 columns (inputs)}. The Arduino scans the keypad by driving rows LOW and checking which column reads LOW.

\begin{center}
\begin{tabular}{|c|c|c|}
\hline
\textbf{Keypad Pin} & \textbf{Arduino Pin} & \textbf{Function} \\
\hline
Row 1 (R1) & D2 & Output \\
Row 2 (R2) & D3 & Output \\
Row 3 (R3) & D4 & Output \\
Row 4 (R4) & D5 & Output \\
Row 5 (R5) & D6 & Output \\
Column 1 (C1) & D7 & Input (Pull-up) \\
Column 2 (C2) & D8 & Input (Pull-up) \\
Column 3 (C3) & D9 & Input (Pull-up) \\
Column 4 (C4) & D10 & Input (Pull-up) \\
Column 5 (C5) & D11 & Input (Pull-up) \\
\hline
\end{tabular}
\end{center}

\subsection{Keypad Operation}
The Arduino scans the keypad using the following steps:
\begin{enumerate}
    \item The \textbf{Arduino sets one row LOW} at a time while keeping others HIGH.
    \item It checks which \textbf{column reads LOW}, indicating a pressed key.
    \item Once detected, the Arduino determines which key was pressed using the \textbf{row-column mapping}.
    \item The \textbf{Shift key (S)} toggles alternate functions, and results are displayed on the LCD.
\end{enumerate}

\section{Keypad Layout}
\subsection{Standard Mode}
\begin{verbatim}
  {'7', '8', '9', '/', 's'},  // s = sin
  {'4', '5', '6', '*', 'c'},  // c = cos
  {'1', '2', '3', ' ', 't'},  // t = tan
  {'0', '.', '=', '+', 'l'},  // l = log
  {'C', 'S', '^', 'R', 'e'}   // S = Shift, R = sqrt, e = exp
\end{verbatim}

\subsection{Shift Mode}
\begin{verbatim}
  {'P', 'E', '!', 'm', 'S'},  // P = pi, E = Euler, ! = factorial, m = mod
  {'a', 'b', 'c', 'd', 'C'},  // Unused characters
  {'s', 'c', 't', ' ', 'I'},  // s = asin, c = acos, t = atan, I = inverse
  {'n', 'L', 'X', '+', 'h'},  // n = ln, L = log10, X = e^x
  {'C', 'S', '^', 'R', 'e'}   // S = Shift, R = sqrt, e = exp
\end{verbatim}




\section{Code Implementation}



The  code for the calculator is available at:  
\url{https://github.com/ellantirohith/EE1003/blob/main/Reports/Calc/codes/main.c}

\section{Conclusion}
This scientific calculator effectively handles standard and advanced functions using a matrix keypad. The shift key extends its capability without requiring extra buttons. 

\end{document}
